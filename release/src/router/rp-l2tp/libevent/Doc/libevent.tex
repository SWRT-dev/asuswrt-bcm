% LIC: GPL
\documentclass{article}
\usepackage{epsfig}
\usepackage[colorlinks]{hyperref}
% Style file for Roaring Penguin software maintenance documents
% Please use these macros in your LaTeX documentation:
% \tclok             -- expands to TCL_OK in fixed-width font
% \tclerr            -- expands to TCL_ERROR in fixed-width font
% \squiggle          -- Outputs "~" (tilde)
% \type{foo}         -- Use for C++ types.  e.g.:  \type{unsigned char}
% \param{bar}        -- Use for function parameters
% \incfile{foo/b.h}  -- Use for C++ header files
% \name{func}        -- Use for function and method names.
\newcommand{\tclok}{\texttt{TCL\_OK}}
\newcommand{\tclerr}{\texttt{TCL\_ERROR}}
\newcommand{\squiggle}{\symbol{"7E}}% \tilde was already taken...
\newcommand{\type}[1]{\textit{#1}}
\newcommand{\param}[1]{\texttt{#1}}
\newcommand{\incfile}[1]{\texttt{#1}}
\newcommand{\name}[1]{\textbf{#1}}

% \function{type}{name}{(arglist)}{description}{returns}
%
% Example:
% \function{int}{abs}{(\type{int} \param{j})}
% {Computes the absolute value of \param{j}.}    % Description
% {The absolute value of \param{j}.}             % Returns
% \begin{itemize}                                % Arguments
% \item \param{j} -- The number whose absolute value is needed
% \end{itemize}
%
% Note that ``arglist'' is typeset in a tabbing environment; you
% can do this to align arguments:
% \function{int}{frob}{(\=\type{int} \param{arg1},\\
% \>\type{char *} \param{arg2}, \\
% \>\type{double} \param{arg3})} ...

\newcommand{\function}[5]{%
  \begin{tabbing}{\rule{\linewidth}{0.5pt}}\\%
    \type{#1} \name{#2}#3\end{tabbing}%
  \begin{description}%
  \item[Description:] {#4}%
  \item[Returns:] {#5}%
  \item[Arguments:]%
  \end{description}%
}
% \method{type}{class}{name}{(arglist)}{description}{returns}
%
% Example:
% \method{void}{Box}{engulf}{(\type{Point} \param{p})}
% {Enlarges \param{self} to include \param{p}}   % Description
% {Nothing.}                                     % Returns
% \begin{itemize}                                % Arguments
% \item \param{p} -- A point to be included in the box.
% \end{itemize}
%
% Note that ``arglist'' is typeset in a tabbing environment as with
% \function

\newcommand{\method}[6]{%
  \begin{tabbing}\rule{\linewidth}{0.5pt}\\%
  \type{#1} \name{#2}::\name{#3}#4\end{tabbing}%
  \begin{description}%
  \item[Description:] {#5}%
  \item[Returns:] {#6}%
  \item[Arguments:] %
  \end{description}%
}

% \simplemethod{type}{class}{name}{(arglist)}{description}
%
% Simpler version when \method is overkill.  Just includes description.
% Example:
% \simplemethod{void}{Box}{engulf}{(\type{Point} \param{p})}
% {Enlarges \param{self} to include \param{p}}   % Description
% Note that ``arglist'' is typeset in a tabbing environment as with
% \function

\newcommand{\simplemethod}[5]{%
  \begin{tabbing}\rule{\linewidth}{0.5pt}\\%
  \type{#1} \name{#2}::\name{#3}#4\end{tabbing}%
  \begin{description}%
  \item[Description:] {#5}%
  \end{description}%
}

% \simplefunction{type}{name}{(arglist)}{description}
%
% Simpler version when \function is overkill.  Just includes description.
% Example:
% \simplefunction{void}{Box}{engulf}{(\type{Point} \param{p})}
% {Enlarges \param{self} to include \param{p}}   % Description
% Note that ``arglist'' is typeset in a tabbing environment as with
% \function

\newcommand{\simplefunction}[4]{%
  \begin{tabbing}\rule{\linewidth}{0.5pt}\\%
  \type{#1} \name{#2}#3\end{tabbing}%
  \begin{description}%
  \item[Description:] {#4}%
  \end{description}%
}

% \tclmethod -- like \method, but for Tcl methods.
\newcommand{\tclmethod}[6]{%
  \begin{tabbing}\rule{\linewidth}{0.5pt}\\%
  \type{#1} \name{#2}.\name{#3} #4\end{tabbing}%
  \begin{description}%
  \item[Description:] {#5}%
  \item[Returns:] {#6}%
  \item[Arguments:]
  \end{description}%
}
% \variable{type}{name}{description}
% Use this for global variables.
%
% Example:
% \variable{int}{Timeout}
% {The timeout value of frobnosticate in seconds.}

\newcommand{\variable}[3]{\rule{\linewidth}{0.5pt}\\%
  \type{#1} \name{#2}%
  \begin{description}%
  \item[Description:] {#3}%
  \end{description}%
}

\newcommand{\synopsis}[3]{\rule{\linewidth}{0.5pt}%
\\%
\begin{tabular}{l@@{ }l@@{}p{4in}}
\name{#1} & \texttt{#2} & \texttt{#3}%
\end{tabular}
\\%
\raisebox{5pt}{\rule{\linewidth}{0.5pt}}}

\setlength{\parindent}{0pt}
\setlength{\parskip}{2pt}

\ifx\pdfoutput\undefined
\newcommand{\eps}{eps}
\else
\newcommand{\eps}{pdf}
\fi

\newcommand{\Le}{\textsf{LibEvent}}
\newcommand{\Es}{\type{EventSelector}}
\newcommand{\Eh}{\type{EventHandler}}

\title{\Le{} Programmers Manual}
\author{David F. Skoll\\\textit{Roaring Penguin Software Inc.}}
\begin{document}
\maketitle

\section{Introduction}
\label{sec:introduction}

Many UNIX programs are event-driven.  They spend most of their time
waiting for an event, such as input from a file descriptor, expiration
of a timer, or a signal, and then react to that event.

The standard UNIX mechanisms for writing event-driven programs are
the \name{select} and \name{poll} system calls, which wait for input
on a set of file descriptors, optionally with a timeout.

While \name{select} and \name{poll} can be used to write event-driven
programs, their calling interface is awkward and their level of
abstraction too low.  \Le{} is built around \name{select}, but
provides a more pleasant interface for programmers.

\Le{} provides the following mechanisms:
\begin{itemize}
\item \textit{Events}, which trigger under user-specified conditions,
  such as readability/writability of a file descriptor or expiration of
  a timer.
\item \textit{Synchronous signal-handling}, which is the ability to
  defer signal-handling to a safe point in the event-handling loop.
\item \textit{Syncronous child cleanup}, which lets you defer calls
  to \name{wait} or \name{waitpid} to a safe point in the event-handling
  loop.
\end{itemize}

\section{Overview}
\label{sec:overview}

Figure~\ref{fig:flow} indicates the overall flow of programs using
\Le{}.
\begin{figure}[htbp]
  \begin{center}
    \epsfig{file=flow.\eps,width=3in}
    \caption{\Le{} Flow}
    \label{fig:flow}
  \end{center}
\end{figure}

\begin{enumerate}
\item Call \name{Event\_CreateSelector} once to create an \emph{Event
    Selector}.  This is an object which manages event dispatch.
\item Open file descriptors as required, and call \name{Event\_CreateHandler}
  to create \emph{Event Handlers} for each descriptor of interest.  You
  can call \name{Event\_CreateTimerHandler} to create timers which are
  not associated with file descriptors.
\item Call \name{Event\_HandleEvent} in a loop.  Presumably, some event
  will cause the program to exit out of the infinite loop (unless the
  program is designed never to exit.)
\end{enumerate}

To use \Le{}, you should \texttt{\#include} the file
\incfile{libevent/event.h}

\section{Types}

\Le{} defines the following types:
\begin{itemize}
\item \Es{} -- a container object which manages event handlers.
\item \Eh{} -- an object which triggers a callback function when an event
  occurs.
\item \type{EventCallbackFunc} -- a prototype for the callback function
  called by an \Eh{}.
\end{itemize}

\section{Basic Functions}
\label{sec:basic-functions}

This section describes the basic \Le{} functions.  Each function is
described in the following format:
\function{type}{name}{(\type{type1} \param{arg1}, \type{type2} \param{arg2})}
{A brief description of the function.  \type{type} is the type of the
return value and \name{name} is the name of the function.}
{What the function returns}
\begin{itemize}
\item \param{arg1} -- A description of the first argument.
\item \param{arg2} -- A description of the second argument, etc.
\end{itemize}

\subsection{Event Selector Creation and Destruction}
\function{EventSelector *}{Event\_CreateSelector}{(\type{void})}
{Creates an \Es{} object and returns a pointer to it.
An \Es{} is an object which keeps track of event handlers.
You should treat it as an opaque type.}
{A pointer to the \Es{}, or NULL if out of memory.}
{None.}

\function{void}{Event\_DestroySelector}{(\type{EventSelector *}\name{es})}
{Destroys an \Es{} and all associated event handlers.}
{Nothing.}
\begin{itemize}
\item \param{es} -- the \Es{} to destroy.
\end{itemize}

\subsection{Event Handler Creation and Destruction}

An \Eh{} is an opaque object which contains information about an
event.  An event may be \emph{triggered} by one or more of three things:

\begin{enumerate}
\item A file descriptor becomes readable.  That is, \name{select} for
  readability would return.
\item A file descriptor becomes writeable.
\item A timeout elapses.
\end{enumerate}

When an event triggers, it calls an event callback function.  An
event callback function looks like this:

\function{void}{functionName}{(\=\type{EventSelector *}\param{es},\\
\>\type{int} \param{fd},\\
\>\type{unsigned int} \param{flags},\\
\>\type{void *}\param{data})}
{Called when an event handler triggers.}
{Nothing}
\begin{itemize}
\item \param{es} -- the \Es{} to which the event handler belongs.
\item \param{fd} -- the file descriptor (if any) associated with the event.
\item \param{flags} -- a bitmask of one or more of the following values:
  \begin{itemize}
  \item \texttt{EVENT\_FLAG\_READABLE} -- the descriptor is readable.
  \item \texttt{EVENT\_FLAG\_WRITEABLE} -- the descriptor is writeable.
  \item \texttt{EVENT\_FLAG\_TIMEOUT} -- a timeout triggered.
  \end{itemize}
\item \param{data} -- an opaque pointer which was passed into
  \name{Event\_AddHandler}.
\end{itemize}

\function{EventHandler *}{Event\_AddHandler}{(\=\type{EventSelector *}\param{es},\\
\>\type{int} \param{fd},\\
\>\type{unsigned int} \param{flags},\\
\>\type{EventCallbackFunc} \param{fn},\\
\>\type{void *}\param{data})}
{Creates an \Eh{} to handle an event.}
{An allocated \Eh{}, or NULL if out of memory.}
\begin{itemize}
\item \param{es} -- the event selector.
\item \param{fd} -- the file descriptor to watch.  \param{fd} must be
  a legal file descriptor for use inside \name{select}.
\item \param{flags} -- a bitmask whose value is one of
  \texttt{EVENT\_FLAG\_READABLE}, \texttt{EVENT\_FLAG\_WRITEABLE} or
  \texttt{EVENT\_FLAG\_READABLE~|~EVENT\_FLAG\_WRITEABLE}.  \param{flags}
  specifies the condition(s) under which to trigger the event.
\item \param{fn} -- the callback function to invoke when the event triggers.
\item \param{data} -- a pointer which is passed unchanged as the last
  parameter of \param{fn} when the event triggers.
\end{itemize}

\function{EventHandler *}{Event\_AddTimerHandler}{(\=\type{EventSelector *}\param{es},\\
\>\type{struct timeval} \param{t},\\
\>\type{EventCallbackFunc} \param{fn},\\
\>\type{void *}\param{data})}
{Creates an \Eh{} to handle a timeout.  After the timeout elapses, the
  callback function is called once only, and then the \Eh{} is automatically
  destroyed.}
{An allocated \Eh{}, or NULL if out of memory.}
\begin{itemize}
\item \param{es} -- the event selector.
\item \param{t} -- the time after which to trigger the event.  \param{t}
  specifies how long \emph{after} the current time to trigger the event.
\item \param{fn} -- the callback function to invoke when the event triggers.
  A timer handler function is always called with its \param{flags} set
  to \texttt{EVENT\_FLAG\_TIMER~|~EVENT\_FLAG\_TIMEOUT}.
\item \param{data} -- a pointer which is passed unchanged as the last
  parameter of \param{fn} when the event triggers.
\end{itemize}

\function{EventHandler *}{Event\_AddHandlerWithTimeout}
{(\=\type{EventSelector *}\param{es},\\
\>\type{int} \param{fd},\\
\>\type{unsigned int} \param{flags},\\
\>\type{struct timeval} \param{t},\\
\>\type{EventCallbackFunc} \param{fn},\\
\>\type{void *}\param{data})}
{Creates an \Eh{} to handle an event.  The event is called when a file
  descriptor is ready or a timeout elapses.  This function may be viewed
  as a combination of \name{Event\_AddHandler} and \name{Event\_AddTimerHandler}.}
{An allocated \Eh{}, or NULL if out of memory.}
\begin{itemize}
\item \param{es} -- the event selector.
\item \param{fd} -- the file descriptor to watch.  \param{fd} must be
  a legal file descriptor for use inside \name{select}.
\item \param{flags} -- a bitmask whose value is one of
  \texttt{EVENT\_FLAG\_READABLE}, \texttt{EVENT\_FLAG\_WRITEABLE} or
  \texttt{EVENT\_FLAG\_READABLE~|~EVENT\_FLAG\_WRITEABLE}.  \param{flags}
  specifies the condition(s) under which to trigger the event.
\item \param{t} -- the time after which to trigger the event.  If the event
  is triggered because of a timeout, the callback function's \param{flags}
  has the \texttt{EVENT\_FLAG\_TIMEOUT} bit set.
\item \param{fn} -- the callback function to invoke when the event triggers.
\item \param{data} -- a pointer which is passed unchanged as the last
  parameter of \param{fn} when the event triggers.
\end{itemize}

\function{int}{Event\_DelHandler}
{(\=\type{EventSelector *}\param{es},\\
\>\type{EventHandler *}\param{eh})}
{Deletes an \Eh{} and frees its memory.  A handler may be deleted from
  inside a handler callback; \Le{} defers the actual deallocation of
  resources to a safe time.}
{0 if the handler was found and deleted, non-zero otherwise.  A non-zero
  return value indicates a critical internal error.}
\begin{itemize}
\item \param{es} -- the event selector which contains \param{eh}.
\item \param{eh} -- the event handler to delete.
\end{itemize}

\subsection{Event Handler Access Functions}

The functions in this section access or modify fields in the
\Eh{} structure.  You should \emph{never} access or modify fields
in an \Eh{} except with these functions.

\function{void}{Event\_ChangeTimeout}
{(\=\type{EventHandler *}\param{eh},\\
  \>\type{struct timeval} \param{t})}
{Changes the timeout of \param{eh} to be \param{t} seconds from now.  If
  \param{eh} was not created with \name{Event\_AddTimerHandler} or
  \name{Event\_AddHandlerWithTimeout}, then this function has no effect.}
{Nothing}
\begin{itemize}
\item \param{eh} -- the \Eh{} whose timeout is to be modified.
\item \param{t} -- new value of timeout, relative to current time.
\end{itemize}

\function{EventCallbackFunc}{Event\_GetCallback}
{(\type{EventHandler *}\param{eh})}
{Returns the callback function associated with \param{eh}.}
{A pointer to the callback function associated with \param{eh}.}
\begin{itemize}
\item \param{eh} -- the \Eh{} whose callback pointer is desired.
\end{itemize}

\function{void *}{Event\_GetData}
{(\type{EventHandler *}\param{eh})}
{Returns the data associated with \param{eh} (the \param{data} argument
  to the \ldots{}AddHandler\ldots{} function.)}
{The data pointer associated with \param{eh}.}
\begin{itemize}
\item \param{eh} -- the \Eh{} whose data pointer is desired.
\end{itemize}

\function{void}{Event\_SetCallbackAndData}
{(\=\type{EventHandler *}\param{eh},\\
  \>\type{EventCallbackFunc} \param{fn},\\
  \>\type{void *}\param{data})}
{Sets the callback function and data associated with \param{eh}.}
{Nothing.}
\begin{itemize}
\item \param{eh} -- the \Eh{} whose callback function and data pointer are
  to be set.
\item \param{fn} -- the new value for the callback function.
\item \param{data} -- the new value for the data pointer.
\end{itemize}

\section{Signal Handling}

In UNIX, signals can arrive asynchronously, and a signal-handler
function may be called at an unsafe time, leading to race conditions.
\Le{} has a mechanism to call a handler function during
\name{Event\_HandleEvent} so that the handler is dispatched just like
any other event handler.  In this way, the signal handler knows that
it is safe to access shared data without interference from another
thread of control.

\Le{} implements this \emph{synchronous signal handling} by setting up
a UNIX pipe, and writing to the write-end inside the asynchronous
handler.  The read end then becomes ready for reading, and triggers
a normal event.  \Le{} encapsulates all the details for you in
two functions.

\function{int}{Event\_HandleSignal}
{(\=\type{EventSelector *}\param{es},\\
  \>\type{int} \param{sig},\\
  \>\type{void (*}\param{handler}\type{)(int} \param{sig}\type{)})}
{Arranges for the function \param{handler} to be called when signal
  \param{sig} is received.  \param{sig} is typically a constant from
  \incfile{signal.h}, such as \texttt{SIGHUP}, \texttt{SIGINT}, etc.
  The \param{handler} function is not called in the context of a UNIX
  signal handler; rather, it is called soon after the signal has been
  received as part of the normal \name{Event\_HandleEvent} loop.

  As a side-effect of calling this function, a UNIX signal handler
  is established for \param{sig}.  Any existing signal disposition is
  forgotten.  If \param{sig} is \texttt{SIGCHLD}, then the
  \texttt{SA\_NOCLDSTOP} flag is set in the \param{struct sigaction} passed
  to the low-level \name{sigaction} function.}
{0 on success; -1 on failure.  Failure is usually due to a UNIX system
  call failing or a lack of memory.}
\begin{itemize}
\item \param{es} -- the event selector.
\item \param{sig} -- the signal we wish to handle.
\item \param{handler} -- the function to call.  It is passed a single
  argument---the signal which is being handled.
\end{itemize}

\function{int}{Event\_HandleChildExit}
{(\=\type{EventSelector *}\param{es},\\
  \>\type{pid\_t} \param{pid},\\
  \>\type{void (*}\param{handler}\type{)(pid\_t} \param{pid}, \type{int} \param{status}, \type{void *}\param{data}\type{)},\\
  \>\type{void *}\param{data})}
{Arranges for \param{handler} to be called when the child process with
  process-ID \param{pid} exits.  \param{pid} must be the return
  value of a successful call to \name{fork}.

  When the process with process-ID \param{pid} exits, \Le{} catches
  the \texttt{SIGCHILD} signal and at some point in the event-handling
  loop, calls \param{handler} with three arguments:  \param{pid} is
  the process-ID of the process which terminated.  \param{status} is
  the exit status as returned by the \name{waitpid} system call.  And
  \param{data} is passed unchanged from the call to \name{Event\_HandleChildExit}.}
{0 on success; -1 on failure.  Failure is the result of lack of memory or
  the failure of a UNIX system call.}
\begin{itemize}
\item \param{es} -- the event selector.
\item \param{pid} -- process-ID of the child process.
\item \param{handler} -- the function to call when the process exits.
\item \param{data} -- a pointer which is passed unchanged to \param{handler}
  when the process exits.
\end{itemize}

\section{Stream-Oriented Functions}

The functions presented in the previous sections are appropriate for
simple events, especially those associated with datagram sockets.  A
higher level of abstraction is required for stream-oriented descriptors.
It would be nice for \Le{} to invoke a callback function when a certain
number of bytes or a specific delimiter have been read from a stream,
or when an entire buffer's worth of data has been written to a stream.

The functions in this section all (unfortunately) have the string
\texttt{Tcp} in their names, because they were originally used with TCP
sockets.  However, they may be used with any stream-oriented sockets,
including UNIX-domain sockets.

All of the stream-oriented functions are built on the simpler event
functions described previously.  They simply add an extra layer of
convenience.  To use the stream-oriented functions,
\texttt{\#include} the file \incfile{libevent/event\_tcp.h}.

\section{Stream-Oriented Data Types}

The stream-oriented functions use the following publicly-accessible type:
\begin{itemize}
\item \type{EventTcpState} -- an opaque object which records the state
  of stream-oriented event handlers.
\end{itemize}

\section{Stream-Oriented Functions}
\label{sec:basic-stream-oriented-functions}

The stream-oriented functions may be broken into two main groups:
Connection establishment, and data transfer.

\subsection{Connection Establishment}

\function{EventHandler *}{EventTcp\_CreateAcceptor}
{(\=\type{EventSelector *}\param{es},\\
  \>\type{int} \param{fd},\\
  \>\type{EventTcpAcceptFunc} \param{f})}
{Creates an event handler to accept incoming connections on the listening
  descriptor \param{fd}.  Each time an incomming connection is accepted,
  the function \param{f} is called.}
{An \Eh{} on success; NULL on failure.}
\begin{itemize}
\item \param{es} -- the event selector.
\item \param{fd} -- a listening socket (i.e., one for which the
  \name{listen}(2) system call has been called.)
\item \param{f} -- a function which is called each time an incoming
  connection is accepted.  The function \param{f} should look like this:

  \type{void} \name{f}{(\type{EventSelector *}\param{es}, \type{int} \param{fd})}

  In this case, \param{es} is the \Es{}, and \param{fd} is the new file
  descriptor returned by \name{accept}(2).
\end{itemize}

\function{void}{EventTcp\_Connect}
{(\=\type{EventSelector *}\param{es},\\
  \>\type{int} \param{fd},\\
  \>\type{struct sockaddr const *}\param{addr},\\
  \>\type{socklen\_t} \param{addrlen},\\
  \>\type{EventTcpConnectFunc} \param{f},\\
  \>\type{int} \param{timeout},\\
  \>\type{void *}\param{data})}
{Attempts to connect the socket \param{fd} to \param{addr} using the
  \name{connect}(2) system call.}
{Nothing.  See below for error-handling notes.}
\begin{itemize}
\item \param{es} -- the event selector.
\item \param{fd} -- a socket which is suitable for passing to
  \name{connect}(2).
\item \param{addr} -- the server address to connect to.
\item \param{addrlen} -- the length of the server address.  The
  three parameters \param{fd}, \param{addr} and \param{addrlen} are passed
  directly to \name{connect}(2).
\item \param{f} -- A function which is called when the connection succeeds
  (or if an error occurs.)  The function \param{f} looks like this:

  \type{void} \name{f}{(\type{EventSelector *}\param{es}, \type{int} \param{fd}, \type{int} \param{flag}, \type{void *}\param{data})}

  The parameters of \param{f} have the following meaning:
  \begin{itemize}
  \item \param{es} -- the event selector.
  \item \param{fd} -- the descriptor.
  \item \param{flag} -- a flag indicating what happened.  It may contain
    one of the following values:
    \begin{itemize}
    \item \param{EVENT\_TCP\_FLAG\_IOERROR} -- the \name{connect} system call
      failed.
    \item \param{EVENT\_TCP\_FLAG\_COMPLETE} -- the \name{connect} system
      call succeeded and the descriptor is now connected.
    \item \param{EVENT\_TCP\_FLAG\_TIMEOUT} -- the \name{connect} system call
      did not complete within the specified timeout.
    \end{itemize}
  \item \param{data} -- a copy of the \param{data} given to
    \name{EventTcp\_Connect}.
  \end{itemize}
\item \param{timeout} -- a timeout value in seconds.  If \name{connect} does
  not complete withing \param{timeout} seconds, the \param{f} is called
  with a flag of \param{EVENT\_TCP\_FLAG\_TIMEOUT}.
\item \param{data} -- an opaque pointer passed unchanged to \param{f}.
\end{itemize}

\subsection{Data Transfer}

There are two stream-oriented functions for data transfer:  One for
reading and one for writing.

\function{EventTcpState *}{EventTcp\_ReadBuf}
{(\=\type{EventSelector *}\param{es},\\
  \>\type{int} \param{fd},\\
  \>\type{int} \param{len},\\
  \>\type{int} \param{delim},\\
  \>\type{EventTcpIOFinishedFunc} \param{f},\\
  \>\type{int} \param{timeout},\\
  \>\type{void *}\param{data})}
{Arranges events to read up to \param{len} characters from the file
  descriptor \param{fd}.  If \param{delim} is non-negative, reading stops
  when the characters \param{delim} is encountered.  After \param{len}
  characters have been read (or \param{delim} has been encountered), or
  after \param{timeout} seconds have elapsed, the function \param{f} is
  called.}
{An \type{EventTcpState} object on success; NULL on failure.  Failure
  is usually due to failure of a UNIX system call or lack of memory.}
\begin{itemize}
\item \param{es} -- the event selector.
\item \param{fd} -- the descriptor to read from.
\item \param{len} -- the maximum number of bytes to read.
\item \param{delim} -- if negative, reading continues until exactly
  \param{len} bytes have been read or the operation times out.  If
  non-negative, reading stops when \param{len} bytes have been read
  or the characters \param{delim} is encountered, whichever comes first.
  Note that supplying a non-negative \param{delim} causes \Le{} to
  invoke the \name{read}(2) system call for \emph{each character}; if
  you are expecting large amounts of data before the delimiter, this
  could be inefficient.
\item \param{f} -- a function which is called when reading has finished,
  an error occurs, or the operation times out.  The function \param{f}
  looks like this:

  \type{void}~\name{f}{(\type{EventSelector~*}\param{es},~\type{int}~\param{fd},~\type{char~*}\param{int~buf},~\type{int}~\param{len},~\type{int}~\param{flag},~\type{void~*}\param{data})}

  The arguments passed to \param{f} are:
  \begin{itemize}
  \item \param{es} -- the event selector.
  \item \param{fd} -- the file descriptor that was passed to
    \name{EventTcp\_ReadBuf}.  If no more activity on \param{fd} is
    required, then you should \name{close} it inside \param{f}.
  \item \param{buf} -- a dynamically-allocated buffer holding the data
    which were read from \param{fd}.  \emph{Do not} free this buffer;
    \Le{} will take care of it.  \emph{Do not} store the pointer value;
    if you need a copy of the data, you must copy the whole buffer.
  \item \param{len} -- the number of bytes actually read from \param{fd}.
  \item \param{flag} -- a flag indicating what happened.  It can have
    one of four values:
    \begin{itemize}
    \item \param{EVENT\_TCP\_FLAG\_COMPLETE} -- the operation completed
      successfully.
    \item \param{EVENT\_TCP\_FLAG\_IOERROR} -- an error occurred during
      a \name{read}(2) or some other system call.
    \item \param{EVENT\_TCP\_FLAG\_EOF} -- EOF was detected before all
      bytes were read.  Nevertheless, \param{len} and \param{buf} have
      valid contents.
    \item \param{EVENT\_TCP\_FLAG\_TIMEOUT} -- the operation timed out
      before all bytes were read.  Nevertheless, \param{len} and
      \param{buf} have valid contents.
    \end{itemize}
  \item \param{data} -- a copy of the \param{data} pointer passed to
    \name{EventTcp\_ReadBuf}.
  \end{itemize}
\item \param{timeout} -- if positive, \Le{} times the operation out
  after \param{timeout} seconds.
\item \param{data} -- an opaque pointer which is passed as-is to
  \param{f}.
\end{itemize}

\function{EventTcpState *}{EventTcp\_WriteBuf}
{(\=\type{EventSelector *}\param{es},\\
  \>\type{int} \param{fd},\\
  \>\type{char *}\param{buf},\\
  \>\type{int} \param{len},\\
  \>\type{EventTcpIOFinishedFunc} \param{f},\\
  \>\type{int} \param{timeout},\\
  \>\type{void *}\param{data})}
{Arranges events to write \param{len} characters from the buffer
  \param{buf} to the file
  descriptor \param{fd}.  After \param{len}
  characters have been written, an error occurs, or
  \param{timeout} seconds have elapsed, the function \param{f} is
  called.}
{An \type{EventTcpState} object on success; NULL on failure.  Failure
  is usually due to failure of a UNIX system call or lack of memory.}
\begin{itemize}
\item \param{es} -- the event selector.
\item \param{fd} -- the descriptor to write to.
\item \param{buf} -- buffer containing characters to write.
  \name{EventTcp\_WriteBuf} allocates its own private copy of the buffer;
  you may free or reuse the buffer once \name{EventTcp\_WriteBuf} returns.
\item \param{len} -- the number of bytes to write.
\item \param{f} -- a function which is called when reading has finished,
  an error occurs, or the operation times out.  The function \param{f}
  is as described in \name{EventTcp\_ReadBuf}.  As a special case,
  you may supply NULL as the value for \param{f}.  In this case,
  \name{EventTcp\_WriteBuf} calls \name{close}(2) on the descriptor
  \param{fd} once writing has finished or timed out, or if an error
  occurs.
\item \param{timeout} -- if positive, \Le{} times the operation out
  after \param{timeout} seconds.
\item \param{data} -- an opaque pointer which is passed as-is to
  \param{f}.
\end{itemize}
\end{document}
